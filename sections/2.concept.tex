\chapter{\concept}
\label{sec:concept}

%목적: 개념을 명확하게, 연구의 범위를 명확하게

%정의
%시나리오 -- 각각의 요소
%각각의 요소에 대한 서포트

본 연구에서는 \concept\이라는 개념을 소개하고 이를 떨어져 사는 가족들에게 제공하는 방법에 대한 탐구를 목표로 한다. \concept\은 함께 사는 사람들은 지속적으로 일상 속에서 느낄 수 있지만, 매우 미묘하고 짧은 순간에 감지하는 것이 어렵기에 이에 대한 명확한 인식이 형성되어 있지 않다. 이전 연구들에서도 사실적이고 풍부한 원거리 인터랙션이나 함께함을 재현하고자 하는 시도가 있었지만, \concept에 대해서는 자세히 탐구된 바가 없었다. 이에 본 장에서는 \concept\을 제공하는 방법을 탐구하기에 앞서 \concept\이라는 개념을 새롭게 정의하고자 한다. 또한 \concept의 개념과 예시 상황들을 소개하고, 본 연구의 범위에 대해 설명한다. 

\section{\concept의 정의}

\wonjung{\concept의 정의가 본 연구에서 제시 (또는 가정)한 것임을 명확하게 해야 하지 않나? 주의 깊게 읽지 않으면 이미 존재하던 개념을 소개하는 것으로 읽을 수도 있을 것 같아서.. }

% 본 절에서는 본 연구에서 탐구하고자 하는 \concept\를 구체적으로 정의한다. 먼저 \concept\ 형성에 중요하게 작용하는 요소들을 제시하고, 예시와 함께 각각의 요소가 어떻게 \concept에 도움을 주는 지 설명한다. \yjc{이어 다양한 함께함과 은근한 함께함의 개념 속에서 본 연구에서 제시하는 개념이 어떻게 해석 될 수 있는지 언급한다.} 

%독서를 좋아하는 부부가 같이 느긋하게 커피를 마시며 각자 책을 볼 때,
%기말고사를 앞둔 딸이 늦은 밤까지 공부하고 그 옆을 어머니가 지켜줄 때,
%공포영화를 보고 싶은 어린 아이가 업무중인 아빠 옆에서 영화를 볼 때

기말고사를 앞둔 딸이 늦은 밤 공부하는 상황을 상상해보자. 그리고 딸의 곁에서 늦은 밤까지 자리를 묵묵히 지켜주며 책을 읽는 어머니를 떠올려보자.딸은 이따금 페이지가 넘어갈 때의 속삭임을 들으며, 함께 있어주는 어머니의 따스한 존재감을 느낀다. 뒤돌아보지 않아도 전해오는 든든함과 편안함 속에서 딸은 공부에 집중한다.
이렇듯, \concept\은 상대방과 함께 있음을 은근하게 인식할 수 있는 상황에서 느끼게 되는 심리적 안정감, 편안함 그리고 함께함을 의미한다.
구체적으로는, 아래 다섯 가지 상황을 동시에 만족할 때 느끼게 되는 함께함의 감정을 말한다. %본 연구에서는 멀리 떨어져 사는 두 사람에게 \concept\을 제공하는 것을 목표로 한다.

%--> 우리가 제시하는 은근한 함께함은 이러한 것이다.
\begin{center}
\begin{minipage}{.6\textwidth}
\begin{enumerate}[label=\Roman*., noitemsep]
	\item 상대방이 나와 같은 공간에 있다고 인지할 수 있는 상태
	\item 상대방의 상황(상태)을 언제든지 파악할 수 있는 상태
	\item 상대방이 나를 의식한다는 것을 인지하고 있는 상태
	\item 언제든지 상대방과 소통할 수 있는 상태
	\item 서로의 활동을 방해하지 않고 있는 상태
\end{enumerate}
\end{minipage}
\end{center}


%scope
\wonjung{아래 문단은 좀 많이 고침. 2번 bullet에서 '현실적으로 보여준다'는 무슨 뜻인지 이해하는 데 실패함.. 문장 자체가 비문이라 고쳐야 함.}
\section{\concept와 함께함}

본 연구에서 추구하는 \concept에 대한 탐구와 기존에 이루어진 함께함(togetherness)에 관한 연구들은 차이가 있다. 함께함의 사전적 의미는 \textit{``다른 사람과 친근하거나 가깝다고 느껴지는 감정(the feeling of being friendly and close with other people)"}이다\cite{def_togetherness}.
함께함은 일상적 상호작용의 매 순간에서 느낄 수 있는 감정이다.
아내가 남편을 간호할 때, 형과 동생이 함께 라면을 끓이는 순간에, 연인끼리 껴안는 순간 등 다양한 상황에 경험할 수 있다. 
이러한 함께함에 관련된 기존의 연구들은 원거리에서 
1) 특정 활동을 함께 할 수 있게 해주거나\cite{flexNfeel}, 
2) 상대방으로 부터 얻을 수 있는 신호(cue)를 최대한 전달하여 상대방을 현실적으로 보여주거나\cite{gibbs1999teleport,orts2016holoportation}을 통해 함께함을 제공하는 것에 집중해 왔다.
%3) 온라인 공간에서의 함께함을 제공하는 것\cite{virtual_togetherness}에 집중해 왔다.
하지만, 이러한 상황에서 경험하게 되는 함께함은 직접적으로 상대방과 같이 있는 느낌을 주는 것을 목표로 한다. 이러한 직접적 함께함은 앞서 소개한 \concept에 비해 일상속에서 꾸준히 지속되도록 하기 어렵다는  차이가 있다.


%% 정의
% \begin{enumerate}
%     \item 상대방이 같은 공간에 있다는 것을 인지
%     \item 상대방의 상황(상태)을 파악
%     \item 상대방이 나를 의식한다는 것을 파악
%     \item 언제든지 소통할 수 있음을 인지
%     \item 서로의 활동을 방해하지 않음 (Low-attentiveness)
% \end{enumerate}
%
% - 상대방에 대한 Pheripheral awarness
%   - 상대방의 물리적인 존재를 느끼고
%   - 상대방이 어떤 상태인지 파악할 수 있고
%   - 상대방에 대한 깊은 이해와 결합하여 배려가 가능한 상태
% - 
% - 긴 시간에 걸쳐 일상속에 녹아들 수 있음

%\section{\concept을 경험하게 되는 상황들}

%\section{은근하지 않은 함께함}

% 은근하지 않은 "함께함"에 관한 시나리오
% 이들은 왜 은근한 함께함에 속하지 않는지?

% \begin{itemize}
%     \item 병간호, 집안일 --- Physical support
%     \item Activity 함께하기: 요리 --- Collaborating on the same activity.
%     \item Intense한 conversation: 밥먹으면서 대화, 잔소리 --- 
%     \item 스킨십 --> 하이파이브
% \end{itemize}

% \section{관련연구}

% %기존의 일과 우리가 제시하는 \concept은 다르며 기존의 연구들에서 제공하고자 했던 함께함은 본 연구의 범위가 아니다.
% %멀리 떨어져 있는 사람들에게 함께함(Togetherness)을 제공하기 위한 연구들은 활발히 진행되어왔다\cite{영재 이것좀 채워줘}.

% \yjc{앞에서부터 두 번 쭉 읽어봤는데, 이 섹션 마지막에 핵심적인 관련 연구를 잠깐 언급하는 게 어떨까요?}\

% co-presence
% peripheral awareness
% virtual co-presence