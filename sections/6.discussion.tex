\chapter{논의}
\label{sec:discussion}

본 장에서는 연구의 한계 및 추후 연구 방향에 대해 서술한다.

\section{로봇 아바타 디자인의 한계}

본 연구에서는 상용 텔레프레즌스 아바타를 사용함에 따라, 아바타 디자인에 대해 깊이 논하지 않았다. 하지만, 아바타의 디자인은 실제 사람과 맡닿는 부분으로서 정교한 탐구가 필요하다.


\subsubsection{외관 디자인}

아바타의 외관은 실험에 지대한 영향을 끼친다. 텔레프레즌스 로봇을 아바타가 아닌 영상 통화 디바이스라고 생각한 실험 참가자는 사용 방식에 있어서 완전히 다른 행태를 보여주었다.

외관과 관련하여 먼저 풀어야 할 문제는 텔레프레즌스 로봇의 디스플레이에 어떤 화면을 띄워야 하는 지 관련 문제이다. 현재 실험은 화면을 끈 상태에서 진행했만, 이 외에도 로봇의 시야를 그대로 보여주기, 사람의 얼굴을 확대해서 보여주기, 프로필 이미지 보여주기 등의 다양한 옵션이 가능하다. 이들 요소의 장점과 단점을 분석하여 적절한 디자인을 선택하는 과정을 거쳐야 한다.

\subsubsection{움직임 디자인}

로봇의 움직임은 곧 상대방의 움직임을 표현한다. 따라서 로봇의 움직임의 정밀함은 곧 상대방에 대한 더 정확한 정보 전달을 의미한다. 움직임 디자인 측면에서 풀어야 하는 문제에는 (1) 사람보다 느린 움직임, (2) 장애물 피해서 움직이기, (3) 로봇의 움직임에 의미를 담는 방법이 있다.


\subsubsection{음성 전달 디자인}

일상 속에서 사용되기 위해서는 음성 전달 방식에 관한 추가적인 탐구가 필요하다. 먼저 사람과 로봇 사이의 거리가 멀어지면 사람의 목소리를 로봇의 마이크로 수집하는 것이 어려워진다. 사용자에게 소형 마이크를 부착하거나, 서비스 사용 환경에 다수의 마이크를 설치하여 사용자의 음성을 수집하는 기술을 개발할 필요가 있다. 두번째로 배경음악이 존재하는 경우, 소리의 전달 자체가 불쾌한 경험을 줄 수 있다. 두 사람 모두 TV를 보는 상황이라면, 같은 채널일 경우 음성 신호 전달의 지연 때문에, 다른 채널이라면 TV 소리가 소음으로서 넘어올 수 있다. 배경소음을 제거하거나 특정 소리만 선택할 수 있는 기술\cite{flanagan1993spatially}이 필요하다.  

%\subsubsection{음성 디자인}

%일상 속에서 사용되기 위해서는 음성 또한 추가적인 연구가 필요하다. 먼저 사람과 로봇 사이의 거리가 멀어지면 사람의 목소리를 로봇의 마이크로 수집하는 것이 어려워진다. 사람에게 부가적인 마이크를 부착하거나, 환경의 여러 마이크를 설치해 사람의 목소리를 수집하는 기술을 연구할 필요가 있다. 두번째로 배경음악이 존재하는 경우, 소리의 전달 자체가 불쾌한 경험을 줄 수 있다. 두 사람 모두 TV를 보는 상황이라면, 같은 채널일 경우 음성 신호 전달의 지연 때문에, 다른 채널이라면 TV 소리가 소음으로서 넘어올 수 있다. 배경소음을 제거하거나 특정 소리만 선택할 수 있는 기술\cite{flanagan1993spatially}이 필요하다.  



\section{실험의 한계 및 확장}

본 연구에서 진행한 실험은 특정한 프로토타입 하나만을 대상으로 하였다. 제시하는 디자인의 장점을 평가하기 위해서는 다른 공간 매핑 방법, 다른 시나리오 상황과 비교하는 실험이 이루어져야 한다. 또한 실험 방식에 있어서도, 한 쌍의 커플에게 오랜 시간 배포하고 그 경과를 보는 장기 사용자 스터디가 필요하다.

먼저, 은근한 함께함을 주는 각 요소에 대해 더 자세한 분석을 하는 과정이 필요하다. 예를 들어 로봇의 배치의 경우, 현재 실험에서는 상대방의 위치와 가장 의미적으로 유사한 곳에 놓았지만, 이 외에도 집에 한가운데 놓거나, 사람을 따라다니게 하는 등의 배치가 가능하다. 디자인의 우수성을 얘기하기 위해서는 다른 매핑 방법, 다른 시나리오 상황과 비교하는 실험이 이루어져야 한다.

또한 실험 방식도 개선해야 한다. 이번 실험은 초기 연구로서, 연구실 환경에서 시나리오 기반으로 실험을 진행했다. 하지만, \concept\는 일상속에서 형성되는 감각으로 실제 환경에 시스템을 배포해보는 과정이 필수적이다. 후속 연구에서는 1주일 간 실제 떨어져 사는 커플을 대상으로 장기간의 사용자 실험을 진행할 예정이다.


