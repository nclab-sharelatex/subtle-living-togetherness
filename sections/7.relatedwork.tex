\chapter{관련 연구}
\label{sec:relatedwork}

\tglee{HomeMeld와 어떻게 align이 되어 있는지 convincing을 시켜주도록 단락별로 앞부분을 더 써라}

\section{떨어져 사는 사람들을 위한 원거리 인터랙션 연구}

\noindent\tglee{이 섹션은 related work 이 너무 많은데, 우선 HomeMeld 에서 cite한 것들 위주로 작성하였음. 필요한 논문이 있으면 계속 추가바람.}

통신 기술와 IoT 및 모바일 기술이 발전하면서 떨어진 사람들을 위한 인터랙션 연구는 활발히 진행되어 왔다. 비디오 채팅과 같은 실시간 대면 커뮤니케이션 서비스들은 오래전부터 많은 사람들이 원격 소통의 채널로 활용했으며 비음성적인 요소를 추가하여 이를 고도화하는 연구도 이루어져 왔다 \cite{judge2010sharing, kirk2010home, neustaedter2012intimacy}. 또한, 텔레프레즌스(Telepresence) 로봇을 활용하여 원격으로 사람의 존재감을 전달하고자 하는 연구도 탐구되었다 \cite{judge2010sharing, yang2017communicating}.
이러한 고도화된 커뮤니케이션 도구들은 더 많은 신호(cue)를 전달해주거나 더 사실적이고 자세한 정보를 재현하고자 노력하였다. 하지만 이러한 방식들은 사용자들의 집중과 노력을 요구하며 커뮤니케이션을 진행하면서 다른 일을 병행하는 것이 어렵다. 특히, 텔레프레즌스 로봇의 경우 사용자가 로봇을 직접 조종해야 하기 때문에 원격 인터랙션에만 오롯이 집중해야 한다.

반면에 몇가지 추상적인 신호들을 멀리 떨어진 집에 전달해 주는 시도들도 탐구되었다. 집안 불빛의 유무와 소리가 들리는 방의 위치 \cite{clark2015haunted} 또는 발소리 \cite{tunnermann2015upstairs} 를 활용하여 함께 사는 느낌을 재현하고자 하였다.
이러한 시도들은 사람의 인기척을 집중을 요구하지 않으면서 전달해 주지만, 너무 정보가 적고 추상적이어서 의미있는 커뮤니케이션으로 연결되지 못한다.
% 사람이 이동할 때에만 정보가 전달되어 상대방의 맥락을 항상 파악하기가 어렵다. 
% 실체감이 없다. 

\section{주변 인지를 활용하여 다른 사람의 맥락을 전달하는 연구}

사람의 주변인지(peripheral awareness) 능력을 활용하여 이를 컴퓨터 인터페이스에 적용하는 방법 또한 여러 연구가 이루어졌다. Ishii et al. 이 소개한 ambientROOM은 사람이 거주하는 물리적인 공간을 정보를 제공하는 인터페이스로써 활용한다 \cite{ishii1998ambientroom}. 저자들은 빛, 소리, 그림자, 시계와 같은 환경적인 요소에 변화를 주어 사용자가 주변인지를 통해 정보를 습득하도록 디자인하였다. 
Pinwheels \cite{ishii2001pinwheels}는 컴퓨터 시스템으로 조절되는 40개의 바람개비를 활용하여 건축물 내부의 정보를 미묘한 움직임과 소리를 통해 제공하는 인터페이스를 제시하였다.
이러한 연구들은 사람의 주변인지가 컴퓨터 시스템에서 활용될 수 있는 가능성을 보여주었지만 간단한 인터페이스 수준에 머물렀다. 전달되는 정보가 매우 단순하며 사람의 컨텍스트에 대한 탐구가 없었기에 이를 통해 다른 사람의 존재를 느끼거나 떨어진 사람들이 같이 사는 듯한 느낌을 주기에는 부족하다.

사람의 주변인지 능력은 매우 적은 집중을 통해서 이루어 지기 때문에 일에 방해받지 않아야 하는 업무환경에서 활용되기에 적합하다. 이에 관련하여 \cite{cadiz1998awareness, dabbish2004controlling}은 물리적으로 떨어진 사람들의 협업을 위한 인식 디스플레이(awareness display)를 제시하였다. 
하지만 인식 디스플레이는 업무 중심의 서비스이기에 연인이나 가족과 같이 가까운 관계인 사람들이 같이 사는 듯한 느낌을 느끼기 보다는 각 사람의 업무상태를 객관적으로 제공하는데에만 집중하였다.

몇몇 연구들은 이러한 주변인지 능력을 활용하여 다른 사람의 상태 또는 존재감을 전달하는 방식에 대해서도 탐구하였다. 대표적으로 Pedersen et al.은 원격으로 두 사람의 존재감을 주변인지를 통해 전달해 주는 AROMA를 제안하였다\cite{pedersen1997aroma}. 그들은 주변 인지에 적합한 추상적 표현(abstract representation) 방식을 디자인 하여, 사용자들의 노력과 집중력을 요구하지 않으면서도 상대방의 상태 변화를 충분히 파악할 수 있는 AROMA 프로토타입을 제작하였다. 이와 비슷하게 Madianou는 주변-코프레즌스(ambient co-presence)에 대해 탐구하였으며 소셜 미디어를 통해 원격으로 친밀감을 조성할 수 있는 방식에 대해 제시하였다 \cite{madianou2016ambient}. 
하지만 이러한 방식들은 상대방의 간단한 상태파악을 통해 친밀감을 형성하는데 그치며 상대방의 행동을 충분히 파악하기에 부족하다. 즉, 두 사람이 서로의 행동에 영향을 주고 받는 수준에 이르지 못하며 상대방과의 즉흥적인 의사소통으로 연결되기에도 어렵다. 또한, 상대방의 상태를 디스플레이나 소셜 미디어를 통해 전달하기에 상대방과 함께 사는 느낌을 만들거나 공간을 공유하는 듯한 느낌을 전해주지 못한다.


\section{물체와의 상호작용에 기반한 실내활동 유추 연구}

본 연구에서 고안한 \sysname는 사람의 상태나 액티비티를 로봇의 실내 위치를 통해 전달한다. 이는 사람의 실내활동이 물체를 사용하거나 바라보는 것으로 어느정도 유추하는 것이 가능하기 때문이며 실제로 \expMapping을 통해 집안에 위치한 사람의 컨텍스트가 가까이 위치한 가구와 강한 상관관계임을 밝혔다.
이와 비슷하게 기존에 몇몇 연구들은 물체 사용(object usage) 정보를 이용하여 머신러닝을 통해 사람의 액티비티를 인식하고자 시도했다 \cite{patterson2005fine, philipose2004inferring, wu2007scalable}.
이를 가족간 인터랙션에 적용한 예로, Kim et al.은 집안 물건을 사용한 흔적을 만들어 공유함으로써 가족간 소셜 프레즌스(social presence)를 유도했다~\cite{kim2017internet}. 
\sysname는 이러한 시도의 연장선에 있으며 단순히 사람의 액티비티를 유추하는 것에 그치지 않고 이를 자연스럽게 상대방에게 전달함으로써 함께 있는 느낌을 재현하였다.