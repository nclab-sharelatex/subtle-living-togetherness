\chapter{\expWorkshop}
\label{sec:design_workshop}

이 장에서는 떨어져 사는 가족들에게 `\concept'\을 제공하는 서비스를 디자인하기에 앞서 진행한 스터디에 대해서 다룬다. 구체적으로, 다른 공간에서 생활하는 두 사람이 \concept\을 느끼기 위해 필요한 디자인 요소들을 탐구를 목적으로 하였다. \concept에 대한 경험과 떨어져 지낸 경험을 가진 사람들을 대상으로 진행한 \expWorkshop 및 그 핵심 결과들을 기술하였다. 이 장에서 얻어낸 결과들을 통해 \ref{sec:system_design}장에서 \concept\을 제공하는 서비스 디자인 요소를 도출한다.


% \yjc{
% 참가자를 왜 6명을 모았고 그런 사람들을 모았는지, 
% 참가자를 어떻게 모집했고 왜 그렇게 했는지,
% 왜 (개인 인터뷰 대신) Group으로 Study를 했는지,
% 왜 1시간 반 동안 했는지, 
% 왜 두 섹션으로 나누어서 진행을 했는지, 
% 각 섹션마다 참가자들에게 어떤 주제를 주고 어떤 질문을 했는지, 
% 왜 그러한 질문을 했는지
% }

%% 방법론에 대한 reasoning 을 하자
%% 왜 그룹인터뷰를 했는지
%% 그룹 인터뷰를 어떠한 방식으로 진행했는지
%% ----------------------------------------
% 스테이지1: 자신이 떨어져 살았던 경험. 실제 현재의 practice. 무엇이 힘들었는지
% 스테이지2-1: 실제 아바타와 함께 산다면, 가장 중요한 요소가 무엇일까?
% 스테이지2-2: 홈멜드 컨셉에 대한 반응.

\section{\expWorkshop\ 개요}

%%%%%%%%%%%% TABLE 시작 %%%%%%%%%%%%%%%%%%%%%%%%%%%%%%%%%%%%%%%%%%%%%%
\begin{table}
\caption{\expWorkshop\ 참가자 정보}
\label{tab:workshop}
\centering
\begin{tabu}{ccccc}
\toprule
\rowfont[c]{\bfseries}
참가자  & 성별  & 직업      & 관계      & 떨어져 산 기간    \\
\midrule
P1      & 남    & 대학생    & 부모-자식 & 53 개월           \\
P2      & 여    & 대학생    & 부모-자식 & 24 개월           \\
P3      & 남    & 대학원생  & 부부      & 6 개월            \\
P4      & 여    & 직장인    & 부부      & -                 \\
P5      & 남    & 대학원생  & 부부      & 44 개월           \\
P6      & 남    & 직장인    & 부모-자식 & 88 개월           \\
\bottomrule
\end{tabu}
\end{table}
%%%%%%%%%%%% TABLE 끝  %%%%%%%%%%%%%%%%%%%%%%%%%%%%%%%%%%%%%%%%%%%%%%

\concept\을 제공하기 위해 필요한 디자인 요소 추출을 위해 6명의 실험참가자를 대상으로 \expWorkshop\을 진행하였다. 실험참가자들은 가족과 함께 산 경험과 떨어져 지낸 경험을 모두 경험한 사람들로 모집하였다. 3명은 떨어져 사는 부부, 나머지 3명은 부모와 떨어져 혼자 사는 사람들로 세부 정보는 표 \ref{tab:workshop}에 기록하였다. \expWorkshop\은 1시간 30분 동안 진행하였으며 각 참가자에게 2만원 상당의 문화상품권을 실험 참가비로 지급하였다. 참가자들은 자신들의 경험을 바탕으로 함께 사는 느낌에 대해 토론하였고 이를 은근하게 전달하는 방법에 대해 논의하였다.
\expWorkshop\을 통해 밝혀낸 핵심 결과는 이후 내용에서 서술한다.


\section{실험 결과}
\wonjung{'흥미로운' quote 한-두가지 추가해주면 좋을듯.}
\subsection{장비 설치 및 착용에 대한 거부감}

실험 참가자들은 모두 가족과 떨어져 지낼 때 음성 및 영상 통화를 사용하였고 이에 대한 불편함을 토로하였다. 특히, 그들은 컴퓨터나 스마트폰을 사용하는 과정이 번거롭다고 지적하였다. 이러한 방법들은 약속을 잡고 장비를 준비해야 하며, 설거지와 같은 집안일을 하면서 같이 진행하는 것이 어렵다고 이야기 하였다. 그들은 멀리 떨어진 가족들과 소통하는데 있어서 직접 장비를 활용하거나 착용하는 것이 큰 불편함을 주고 자신의 일상을 공유하는 것을 방해한다는 것에 동의하였다.

\subsection{사람의 실내 위치}

몇몇 참가자들은 상대방의 집안에서의 위치를 보여주는 것이 \concept\을 제공하는데 효과적일 것이라고 주장하였다. 그들은 함께 사는 느낌을 재현하기 위해서는 상대방이 현재 무엇을 하고 있는지를 아는 것이 필요하다고 언급하며, 이 때 상대방의 위치를 전달해 주면 쉽게 상대방의 행동을 알 수 있을 것이라고 하였다. 예를 들어, 상대방이 냉장고 앞에 있다는 것을 알면, 상대방이 음식이나 음료를 찾는 중이라는 것을 쉽게 유추할 수 있다.

또한, 참가자들은 상대방의 위치를 `\location'에 직접 표시하는 것이 사용자로 하여금 자연스럽게 상대방의 정보를 받아들일 수 있는 방법임에 동의하였다. 그들은 이러한 정보를 거슬리지 않게 전달되는 것이 중요하다고 강조했다. 예를 들어, 상대방의 정보를 음성으로 전달해 주거나 글의 형태로 보여주는 것은 너무 피곤하고 불편할 것이라고 이야기 했다. 반면, \location\를 재현하는 것은 사용자가 집안을 스치듯이 보아도 쉽게 상대방의 존재를 느낄 수 있고 자신이 하는 일에 방해받지 않을 것이라고 언급했다.

\subsection{사람이 바라보는 방향}
참가자들은 실내 위치를 언급한 이후 사람이 바라보는 방향에 대해서도 논의하였다. 몇몇 참가자들은 어떤 사람의 응시 방향은 응시 대상과 인터랙션을 하고자 하는 의지가 드러나는 신호라고 언급하였다. 때문에 이를 원격의 상대방이 인지할 수 있도록 전달해 주는 것이 중요하다고 언급했다. 또한, 참가자들은 사람이 바라보는 대상은 그 사람이 무엇에 집중하고 있는지를 드러내며 실내 위치와 더불어 어떤 행동을 하는지를 더 명확하게 보여줄 수 있을 것이라고 주장했다.

\subsection{실체감을 위한 유형물의 필요}

참가자들은 \concept\을 제공하는데 있어 유형(有形)의 물체를 사용하는 것이 중요하다는 것을 지목하였다.
% P3
홀로그램이나 가상현실(virtual reality) 및 증강현실(augmented reality) 기술을 활용하는 경우 가상의 물체가 사용자의 몸을 통과하며 지나가거나 손을 뻗었을 때 만져지지 않는데, 이러한 점이 실존감을 떨어뜨리는 요인이 될 것이라고 언급하였다.
이는 기존 연구에서도 밝혀진 바 있다 \cite{azmandian2016haptic}.

또한, 그들은 이러한 유형체가 물리적으로 공간을 차지하는 것이 \concept\을 형성하는데 도움이 될 것이라고 강조했다. 실제로 두 사람이 같은 공간에서 사는 것처럼 부딪히지 않기 위해 자리를 비켜주거나 상대방이 화장실을 사용하는 동안 기다려야 하는 상황을 재현할 수도 있다.