\chapter{결론}
\label{sec:conclusion}

본 연구는 학업, 직장으로 어쩔 수 없이 떨어져 사는 가족을 위해 함께 삶이 제공하는 다양한 감각 중 일상 속에서 지속적으로 상대방의 존재를 느낄 수 있을 때 형성되는 \emph{은근한 함께 삶}이라는 감각에 주목했다. 또한 이를 원거리에서 실현하기 위한 방법으로 \emph{로봇 아바타 기반 항시적 위치 동기화}를 제시했다. 이어 동기화를 위한 핵심 기술로서 \emph{가구를 중심으로 한 의미적 위치 및 방향 재현 기술}을 고안했다. 제안하는 방식이 은근한 함께함에 미치는 효과를 확인하기 위해 5쌍의 실제 부부를 대상으로 시나리오 중심 사용자 스터디을 진행했고, 이로부터 상대방의 위치에서 들려오는 목소리, 대화의 공백에 대한 적은 부담, 로봇의 움직임에 따른 대화 기회 제공에 대한 긍정적인 반응을 관찰했다.

본 연구의 의의는 다음과 같다.
먼저, 소셜 컴퓨팅 분야에 `\concept'\이라는 중요한 개념을 새롭게 제시한다. 이는 떨어져 사는 가족들에게 적합한 새로운 장르의 원거리 인터랙션 기술 및 서비스의 지평을 연다.
둘째, 떨어져 사는 가족들에게 \concept\을 효과적으로 제공할 수 있는 \approach\을 고안하였다.
셋째, \concept에 대한 실험의 어려움을 정리, 이를 해결하기 위한 초기 실험 디자인을 제시한다.
마지막으로, 인간참여형 \sysname\를 만들고 제한적인 실험환경에서 \concept의 효과를 탐구하였다.

함께 삶은 모든 인간에게 필요한 필수적인 감각이다. 해당 연구가 기반이 되어 떨어져 사는 수많은 가족들에게 새로운 가능성을 열어줄 수 있길 바란다.