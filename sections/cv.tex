%%
%% 약력 시작
%% Curriculum Vitae
%% 약력 작성은 선택사항임. 약력 내용도 알맞게 바꿀 수 있음
% @command curriculumvitae 이력서
% @options [1 | 2 | 3 |4 ]
% - 1 : 본문과 약력이 둘 다 한글일 때  | 2 : 본문은 한글인데 약력이 영어일 때 | 3 :  본문과 약력이 둘 다 영어일 때  | 4 : 본문은 영어인데 약력이 한글일 때 

\curriculumvitae[1]

    % @environment personaldata 개인정보
    % @command     name         이름
    %              dateofbirth  생년월일
    %              birthplace   출생지
    %              domicile     본적지
    %              address      주소지
    %              email        E-mail 주소
    % - 위 6개의 기본 필드 중에 이력서에 적고 싶은 정보를 입력
    % input data only you want
    \begin{personaldata}
        \name       {안 진 현}
        \dateofbirth{199x}{x}{xx}
        \birthplace {...}
        \address    { ...}
     \end{personaldata}

    % @environment education 학력
    % @options [default: (none)] - 수학기간을 입력
    \begin{education}
        \item[2007. 3.\ --\ 2009. 2.] 고등학교 (2년 수료)
        \item[2009. 2.\ --\ 2013. 8.] 한국과학기술원 수리과학과 (학사)
        \item[2013. 9.\ --\ 2016. 2.] 한국과학기술원 수리과학과 (석사)
    \end{education}

    % @environment career 경력
    % @options [default: (none)] - 해당기간을 입력
    \begin{career}
        \item[2013. 9.\ --\ 2016. 2.] 한국과학기술원 수리과학과 일반조교
    \end{career}

    % @environment activity 학회활동
    % @options [default: (none)] - 활동내용을 입력
%%    \begin{activity}
%%        \item J. Choi, \textbf{Yong-Hyun Kim}, K.J. Chang, and D. Tomanek,
%%             \textit{Occurrence of itinerant ferromagnetism in C/BN superlattice
%%             nanotubes}, 5th Asian Workshop on First-Principles Electronic
%%             Structure Calculations, Seoul (Korea), October., 2002.
%%    \end{activity}
%% 학회활동을 쓰고싶으시면, 이 문서와 클래스 문서의 학회활동 부분을 사용하십시오.

    % @environment publication 연구업적
    % @options [default: (none)] - 출판내용을 입력
    \begin{publication}
        \item J. Ahn, \textit{Analysis of Tail Probability of Interference at a Node in 2-dimensional Homogeneous Poisson Point Process}, Master Thesis, Korea Adv. Inst. Science, Techn., Daejeon, Republic of Korea, 2016.
    \end{publication}