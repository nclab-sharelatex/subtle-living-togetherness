% 논문 서지, 초록, 핵심 낱말, 영문 초록, 영어 핵심 낱말 (Information of thesis, abstract in korean, keywords in korean, abstract in english, keywords in english)
% 한글 초록은 500자를, 영문 초록은 300 낱말을 넘지 않아야 함
% 핵심 낱말은 5 개 이내로 넣음
% 한글 초록에 영문 글자를 쓰지 않도록 한다.

\begin{summary}      
학업, 직장으로 어쩔 수 없이 떨어져 사는 가족들은 영상통화, SNS, 메신저 등의 첨단 연결기술에 삶의 많은 부분을 기댄다. 하지만 이들 기술의 높은 완성도와 가용성에도, 현재 이들의 삶은 함께 살아가는 삶과 비교하면 아직 많이 부족하다. 나는 함께 삶이 제공하는 다양한 감각 중 일상 속에서 지속적으로 상대방의 존재를 느낄 수 있을 때 형성되는 \emph{은근한 함께 삶}이라는 감각에 주목한다. 또한 이를 원거리에서 실현하기 위한 방법으로 \emph{로봇 아바타 기반 항시적 위치 동기화}를 제시한다. 이어 동기화를 위한 핵심 기술로서 \emph{가구를 중심으로 한 의미적 위치 및 방향 재현 기술}을 고안했다. 제안하는 방식이 은근한 함께함에 미치는 효과를 확인하기 위해 5쌍의 실제 부부를 대상으로 시나리오 중심 사용자 스터디을 진행했다. 우리는 이로부터 상대방의 위치에서 들려오는 목소리, 대화의 공백에 대한 적은 부담, 로봇의 움직임에 따른 대화 기회 제공에 대한 긍정적인 반응을 관찰했다. 본 논문은 \emph{은근한 함께 삶}을 위한 새로운 장르의 원거리 인터랙션 기술 및 서비스의 초기 연구로서 향후 관련 연구에 기초 자료로 활용될 수 있다.
\end{summary}

\begin{Korkeyword}
은근한 함께 삶, 로봇 아바타, 떨어져 사는 가족, 소셜 컴퓨팅
\end{Korkeyword}

%%%%%%%%%%%%%%%%%%%%%%%%%%%%%%%%%%%%%%%%%%%%%%

\begin{abstract}
Families living apart relies heavily on cutting-edge communication tools such as video calls, social network services, and messengers. However, despite the high availability and accessibility of such services, separated families still suffer from subtle voids which make the experience incomparable to the those of living together. Here I focus on exploring \emph{subtle living-togetherness}, a sense of togetherness formed when we can feel the other's presence consistently in everyday lives. We propose a \emph{robot avatar based continuous location synchronization} to support the subtle living-togetherness experience. To realize the synchronization, we designed a functional-object based semantic location mapping algorithm. We conducted a scenario-based user study with 5 couples in a lab environment and observed that our design considerations positively affected to the subtle living-togetherness. The paper concludes by discussing the implications for opening up this new genre of remote communication systems.
\end{abstract}
 
\begin{Engkeyword}
Subtle living-togetherness, Robotic avatar, Families living apart, Social Computing
\end{Engkeyword}
