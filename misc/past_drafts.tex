\chapter{서론}

18/06/19
%%%%%%%%%%%%%%%%%%%%%%%%%%%%%%%%%%%%%%%%%%%%%%%%%%%%%%%%
% 지구촌 -- 사람간 연결 기술의 발전
1964년 마셜 맥루헌이 지구촌이라는 말을 꺼낸 이후로\cite{mcluhan1994understanding}, 우리는 지구 반대편에서도 메신저, 영상통화, 소셜 미디어, 인터넷을 통해 서로 같이 얘기하고, 토론할 수 있게 되었다.
% 아쉬움 -- 어쩔 수 없이 떨어져 사는 사람들
그러나 이러한 사람간 연결기술의 발전에도 아직 아쉬움을 느끼는 사람들이 있다. 직장 때문에, 학업 때문에 어쩔 수 없이 떨어져 살아가는 가족들이다. 미국에만 360만명의 부부가 떨어져 살고 있으며, 영국의 경우 전체 커플의 10\%가 서로 떨어져 산다\,\cite{strohm2009living, duncan2013people}. 
한국도 크게 다르지 않다. 비동거 맞벌이 가구 수는 전체 가구의 4.6\%에 달하며, 서울의 경우 10쌍 중 1쌍이 기러기 가족 생활을 하고 있다\,\cite{rock2016goose, wise2012seoul}.
% 함께 사는 듯한 경험
우리는 조금 더 큰 꿈을 꾸어보려고 한다. 과연 이들에게 떨어져 있음에도 마치 함께 사는 듯한 경험을 느끼게 만들어 줄 수 있을까?

%%%%%%%%%%%%%%%%%%%%%%%%%%%%%%%%%%%%%%%%%%%%%%%%%%%%%%%%
