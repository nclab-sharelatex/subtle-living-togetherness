%%
%% 참고문헌 시작
%% bibliography
%% 위는 보기이므로 학과 또는 논문의 특성에 맞게 조정 가능함. 다만, 참고문헌마다 충분한 정보가 들어 있어야 한다.
\begin{thebibliography}{00}

\bibitem{FD1} 박상우, \underline{동시 송수신 안테나를 두 개 쓰는 협력 인지 무선통신망에 알맞은 전 이중 통신}, 한국과학기술원 석사 학위 논문, 2016.

\bibitem{RVP1} 송익호, 박철훈, 김광순, 박소령, \underline{확률변수와 확률과정}, 자유아카데미, 2014.

\bibitem{ML1} 송익호, 안태훈, 민황기, \underline{인지 무선에서의 광대역 주파수 검출 방법 및 장치}, 특허등록번호 10-1494966, 2015년 2월 12일.

\bibitem{SOCA1} 호우위시, 이원주, 이승원, 안태훈, 이선영, 민황기, 송익호, “선형 판별 분석에서 부류안 분산 행렬의 영 공간 재공식화,” \underline{한국통신학회 2012년도 추계종합학술발표회}, 대한민국 고려대학교, 242-243쪽, 2012년 11월.

\bibitem{EF1} 민황기, 안태훈, 이승원, 이성로, 송익호, “비간섭 전력 부하 감시용 고차 적률 특징을 갖는 전력 신호 인식,” \underline{한국통신학회논문지}, 제39C권, 제7호, 608-614쪽, 2014년 7월.



\bibitem{FD2} S. Park, \textit{Full-Duplex Communication for Cooperative Cognitive Radio Networks with Two Simultaneous Transmit and Receive Antennas}, Master Thesis, Korea Adv. Inst. Science, Techn., Daejeon, Republic of Korea, 2016.

\bibitem{RVP2}  I. Song, J. Bae, and S. Y. Kim, \textit{Advanced Theory of Signal Detection: Weak Signal Detection in Generalized Observations}, Springer-Verlag, 2002.

\bibitem{ML2} I. Song, T. An, and J. Oh, \textit{Near ML decoding method based on metric-first search and branch length threshold,} registration no. US 8018828 B2, Sep. 13, 2011, USA. 

\bibitem{SOCA2} H.-K. Min, T. An, S. Lee, and I. Song, “Non-intrusive appliance load monitoring with feature extraction from higher order moments,” in \textit{Proc. 6th IEEE Int. Conf. Service Oriented Computing, Appl.,} Kauai, HI, USA, pp. 348-350, Dec. 2013.

\bibitem{EF2} I. Song and S. Lee, “Explicit formulae for product moments of multivariate Gaussian random variables,” \textit{Statistics, Probability Lett.,} vol. 100, pp. 27-34, May 2015.

\end{thebibliography}